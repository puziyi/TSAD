% Chapter 5

\chapter{Conclusion and Future Work} % Main chapter title

\label{Chapter5} % For referencing the chapter elsewhere, use \ref{Chapter1} 

%----------------------------------------------------------------------------------------

\section{Conclusion}
In our project, we first illustrate different existing research on TCN model, and find the characteristics in it. Then we checked predict ability of TCN over other relatively traditional sequential models, such as Gated recurrent units (GRUs), long short-term memory (LSTM), vanilla RNN and False nearest neighbor (FNN),  by using some specific sequence datas. And in order to have a better result, we tune the hyper parameters in TCN model. After comparison, TCN showed itself as both robust and accuracy even amongst deep learning models. However, it can not demonstrated convincing performance. Next we try to discover the data sensitivity of TCN model and add LSTM model as comparison. We find that TCN performs better than LSTM in fluctuated data. And for short data size, TCN need more time to attain a good result, while for long and huge data, TCN have ability to perform better.

\section{Further work}
However, our group still found some limitations and hope to do more improvements on following aspects:
\begin{itemize}
    \item Limited dataset. In project we tried several different datasets and choose Gefcom2014 electronic data as object dataset. The Gefcom2014 data have some good features such as regular and smooth. However, it also bring limitation to model fitting and make the experiment not that convincing. And we also need more type of data to verify the characteristics we found in TCN. So it's better for us to try more different time series data and test the finding of our project. 
    \item zone different
\end{itemize}