% Chapter 1

\chapter{Introduction to Time Series and Sequential Modeling} % Main chapter title

\label{Chapter1} % For referencing the chapter elsewhere, use \ref{Chapter1} 

%----------------------------------------------------------------------------------------

% Define some commands to keep the formatting separated from the content 
\newcommand{\keyword}[1]{\textbf{#1}}
\newcommand{\tabhead}[1]{\textbf{#1}}
\newcommand{\code}[1]{\texttt{#1}}
\newcommand{\file}[1]{\texttt{\bfseries#1}}
\newcommand{\option}[1]{\texttt{\itshape#1}}

%----------------------------------------------------------------------------------------

\section{What Is A Time Series}

A time series is a sequential set of data points, measured typically over successive times
% defination
\subsection{Categories}
\begin{itemize}
    \item univariate vs. multivariate. A time series containing records of a single variable is termed as univariate, but if records of more than one variable are considered then it is termed as multivariate.
    \item linear vs. non-linear. A time series model is said to be linear or non-linear depending on whether the current value of the series is a linear or non-linear function of past observations.
    \item discrete vs. continuous. In a continuous time series observations are measured at every instance of time, whereas a discrete time series contains
    observations measured at discrete points in time.
\end{itemize}
\subsection{Components of a Time Series}
\begin{itemize}
    \item Trend
    \item Seasonal variation
    \item Cyclical variation
    \item Irregular variation
\end{itemize}

\section{How to identify time series anomaly period}

Time series anomaly detection can be transformed into a task where the goal is to model the time series; and given this model, it finds periods where the predicted values are significantly different from the others. Traditionally, researchers use ARMA, ARIMA, GARCH, and other statistics-based methods to do modeling tasks.
% How to define anomaly and how to evaluate such a judgement?
\section{Statistical Methods and Comparassion}
TODO: Stationarity and Whithe noise
\subsection{AR and MA}
TODO: Details
\subsection{ARMA and ARIMA}
TODO: Details
\subsection{GARCH}
TODO: Details

%----------------------------------------------------------------------------------------

\section{Architecture of This Report}

\begin{itemize}
\item Chapter 1: Introduction to sequential modeling and time series anomaly detection
\item Chapter 2: Deep Learning theory and Related Derivative
\item Chapter 3: Experiment 1: TCN in various prediction issues
\item Chapter 4: Experiment 2: Time series anomaly detection in grid load issue
\item Chapter 5: Conclusion and future directions
\end{itemize}
