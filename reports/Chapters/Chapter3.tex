% Chapter 3

\chapter{Experiment 1} % Main chapter title

\label{Chapter3} % For referencing the chapter elsewhere, use \ref{Chapter1} 

%----------------------------------------------------------------------------------------

In this report, we denote our experiment of reproducing existed works, including Shaojie Bai(2018), Alberto Gasparin(2019), as "Experiment 1". And the reasons to do such a reimplement are list following.

As we addressed in \ref{Chapter1}, there are many kinds of data have the feature successive in time in common, and scholars from different research area contributed multiple views, methods, and tools to do modeling; so it is quite massive for us to have a concise plan for further work if we go through papers one by one. Instead, we will use serval mainstream models(that are introducted in \ref{Chapter2}) and apply them to different kinds of data sets. And we also want to find an ideal data set for further experiments and implement our own research framework through such a simple reproducing and comparison. 

\section{Programming Environment}
We will use Python3 with Pandas, Numpy, and TensorFlow packages for all experiments in this project. And we use CityU Burgundy HPC as part of our hardware supports.

\section{Data Processing}

\subsection{Sequential MNIST}
\subsection{JSB Chorales, Music}
\subsection{PennTreebank, LM}
\subsection{Dow Jones Industrial Average, Finance}

\section{Code Design}


\section{Numeric results}
